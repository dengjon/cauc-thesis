%--------------字体------------------
\newcommand{\FakeBoldSize}{4}

%%%% 定义字体族简称
% 一般的字体
\setCJKfamilyfont{songti}{SimSun.ttf}[Path=fonts/]
\setCJKfamilyfont{heiti}{SimHei.ttf}[Path=fonts/]
\setCJKfamilyfont{kaiti}{KaiTi.ttf}[Path=fonts/]
\setCJKfamilyfont{lishu}{LiSu.ttf}[Path=fonts/]
% 粗体
\setCJKfamilyfont{fakesong}{SimSun.ttf}[Path=fonts/, AutoFakeBold={\FakeBoldSize}]
\setCJKfamilyfont{fakehei}{SimHei.ttf}[Path=fonts/, AutoFakeBold={\FakeBoldSize}]
\setCJKfamilyfont{fakeli}{LiSu.ttf}[Path=fonts/, AutoFakeBold={\FakeBoldSize}]

%%%% 设置全文中英文主字体
\setCJKmainfont[Path=fonts/]{SimSun.ttf}
\setmainfont{Times New Roman} % 英文主字体

%%%% 创建字体格式设置命令
% 伪粗体
\newcommand{\fakehei}{\CJKfamily{fakehei}}
\newcommand{\fakesong}{\CJKfamily{fakesong}}
\newcommand{\fakeli}{\CJKfamily{fakeli}}
% 正常体
\renewcommand{\songti}{\CJKfamily{songti}}
\renewcommand{\heiti}{\CJKfamily{heiti}}

\newcommand{\chuhao}{\fontsize{42pt}{\baselineskip}\selectfont} % 初号
\newcommand{\sanhao}{\fontsize{15.75pt}{\baselineskip}\selectfont} % 三号
\newcommand{\sihao}{\fontsize{14pt}{\baselineskip}\selectfont} % 四号
\newcommand{\xiaosi}{\fontsize{12pt}{\baselineskip}\selectfont} % 小四
\newcommand{\wuhao}{\fontsize{10.5pt}{\baselineskip}\selectfont} % 五号
\newcommand{\xiaowu}{\fontsize{9pt}{\baselineskip}\selectfont} % 小五
\newcommand{\timu}{\fontsize{16pt}{\baselineskip}\selectfont\CJKfamily{heiti}} % 题目格式
\newcommand{\zhaiyao}{\fontsize{16pt}{\baselineskip}\selectfont\CJKfamily{lishu}} % 摘要格式

%%%% 目录字体
\titlecontents{chapter}[0pt]{\heiti\zihao{-4}}{\thecontentslabel\quad}{}
        {\titlerule*[10pt]{$\ldots$}\contentspage}

\titlecontents{section}[0pt]{\songti\zihao{-4}}{\quad\quad\thecontentslabel\quad}{}
        {\titlerule*[10pt]{$\ldots$}\contentspage}
